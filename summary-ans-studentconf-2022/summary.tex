\documentclass{anstrans}
%%%%%%%%%%%%%%%%%%%%%%%%%%%%%%%%%%%
\title{Energy alternatives to reduce the University of Illinois at Urbana-Champaign campus CO$_2$ emissions}
\author{Roberto E. Fairhurst Agosta, Tomasz Kozlowski}

\institute{
University of Illinois at Urbana-Champaign, Dept. of Nuclear, Plasma, and Radiological Engineering\\
ref3@illinois.edu
}

%%%% packages and definitions (optional)
\usepackage{graphicx} % allows inclusion of graphics
\usepackage{booktabs} % nice rules (thick lines) for tables
\usepackage{microtype} % improves typography for PDF
\usepackage{xspace}
\usepackage{tabularx}
\usepackage{caption}
\usepackage{floatrow}
\usepackage{subcaption}
\usepackage{enumitem}
\usepackage{placeins}
\usepackage{amsmath}
\usepackage[acronym,toc]{glossaries}
\include{acros}
% \makeglossaries

\usepackage[printwatermark]{xwatermark}
\usepackage{xcolor}
\usepackage{graphicx}
\usepackage{lipsum}


\renewcommand{\vec}[1]{\bm{#1}} %vector is bold italic
\newcommand{\vd}{\bm{\cdot}} % slightly bold vector dot
\newcommand{\grad}{\vec{\nabla}} % gradient
\newcommand{\ud}{\mathop{}\!\mathrm{d}} % upright derivative symbol

\newcolumntype{c}{>{\hsize=.56\hsize}X}
\newcolumntype{b}{>{\hsize=.7\hsize}X}
\newcolumntype{s}{>{\hsize=.74\hsize}X}
\newcolumntype{f}{>{\hsize=.1\hsize}X}
\newcolumntype{a}{>{\hsize=.45\hsize}X}
%\usepackage[pagestyles]{titlesec}
%\titleformat*{\subsection}{\normalfont}
%\titleformat{\section}{\bfseries}{Item \thesection.\ }{0pt}{}

%\newwatermark[allpages,color=gray!50,angle=45,scale=3,xpos=0,ypos=0]{DRAFT}

\begin{document}
%%%%%%%%%%%%%%%%%%%%%%%%%%%%%%%%%%%%%%%%%%%%%%%%%%%%%%%%%%%%%%%%%%%%%%%%%%%%%%%%

%Outline

%Introduction
%%Motivation
%%Objectives

%Storage Mechanisms

%Methodology: How to calculate everything
%%Scenarios/Dispatch models

%Results

%Conclusions


\section{Introduction}

% Motivation: Why is decarbonization necessary
Energy is one of the largest contributors to economic growth.
In the future, economies and populations will continue to expand, and their energy demand will accompany such a change.
Meeting these future needs requires the development of clean energy sources to ease the increasing environmental concerns.
In 2019, electricity generation was one of the economic sectors that released the most \glspl{GHG} in the US \cite{us_epa_sources_2015}.
Decarbonizing electricity generation will allow us to meet the increases in energy demand and address the environmental concerns simultaneously.

\begin{figure}[htbp!] %or H 
    \centering
    \includegraphics[width=0.90\linewidth]{figures/elec-distrib}
    \hfill
    \caption{Distribution of electricity generation of UIUC campus in fiscal year 2019 \cite{isee_illinois_2020}.}
    \label{fig:elec-distrib}
\end{figure}

% UIUC
% This work focuses on the decarbonization of UIUC campus grid, aligning its objectives with the Illinois Climate Action Plan (iCAP) \cite{institute_for_sustainability_energy_and_environment_illinois_2015, institute_for_sustainability_energy_and_environment_illinois_2020}.
% In 2008, UIUC signed the American College and University Presidents’ Climate Commitment, formally committing to becoming carbon neutral as soon as possible, no later than 2050.
% The university developed the iCAP in 2010 as a comprehensive roadmap toward a sustainable campus environment.

This work studies several energy alternatives for decarbonizing the \gls{UIUC} campus grid.
The alternatives presented here can be extrapolated to any other grid.
We focus on the \gls{UIUC} campus grid because it is a well-characterized grid integrating a diverse mix of energy sources, as shown in Figure \ref{fig:elec-distrib}.
% Campus Grid
%% Abbott
The Abbott Power Plant is a co-generation facility that supplies to the campus with electricity and steam.
The plant has multiple gas turbines and boilers that run on natural gas, fuel oil or coal.
The production of high-pressure steam spins a turbine to drive a generator and produce electricity.
The low-pressure exhaust steam fulfills the space heating, water heating, and space cooling requirements from campus.
Abbott’s maximum capacity is 85 MW for electricity and 800 klbs/h for steam \cite{uiucfs_abbott_nodate}.
%% Purchased
With the goal of providing cost effective utilities, the university purchases electricity with the assistance of a market advisor \cite{uiucfs_energy_2015}.
%% Wind
The Rail Splitter Wind Farm has 67 wind turbines of 1.5 MW each, providing the farm a maximum power output of 100.5 MW.
Due to a power purchase agreement, the university buys an 8.6\% of the total wind generation from the farm \cite{rail_splitter_illinois_2016, uiucfs_energy_2015}.
%% Solar
The Solar Farm 1.0 comprises 18,867 modules, which cover a land area of 20.8 acres and produce a power output of 4.68 MW \cite{uiucfs_solar_2017}.
The small scale solar generation comprises all the solar generation distributed around campus.
This work considers the small scale solar component to be negligible.

% Objectives
The main objective of this work is to analyze multiple electricity generation alternatives for decreasing CO$_2$ emissions on UIUC campus.
For this reason, this paper evaluates two scenarios increasing wind and solar generation capacity.
The second scenario also considers the addition of a nuclear reactor into the mix.
Due to the non-dispatchable nature of wind and solar generation, both scenarios incorporate electricity storage mechanisms.


\section{Electricity Storage Mechanisms}

This work considers two storage mechanisms, Li-ion batteries and hydrogen produced via electrolysis of water.
This section briefly describes each of them and their integration into the grid.

% Li-ion
The first prototype of a Li-ion battery was developed in 1985, becoming recently popular in the last two decades.
Li-ion batteries are rechargeable and are commonly used in laptops and cellphones, electric vehicles, Uninterruptible Power Supplies (UPS), and for applications such as computers, communication technology, and medical technology.
A high energy density and low self-discharge characterize this type of battery.
Additionally, it has a charge-discharge efficiency of 80-90\%, making it a good candidate for electricity storage \cite{sun_car_2010}.
This work considers the charge-discharge efficiency of Li-ion batteries to be 85\%.

% Hydrogen
The electrolysis of water is a well-known process whose commercial use began in 1890.
This process produces approximately 4\% of the worldwide H$_2$.
% The process is ecologically clean because it does not emit GHGs.
% However, in comparison with other methods, electrolysis is a highly energy-demanding technology.
Three electrolysis technologies exist.
Alkaline-based is the most common, the most developed, the lowest in capital cost, and the lowest in efficiency.
Proton exchange membrane (PEM) electrolyzers are more efficient but more expensive than Alkaline electrolyzers.
Solid Oxide Electrolysis Cells (SOEC) electrolyzers are the most electrically efficient but the least developed.
% SOEC technology has challenges with corrosion, seals, thermal cycling, and chrome migration.
The first two technologies work with liquid water, and the latter requires high-temperature steam, so this work refers to the first two as \gls{LTE} and the latter as \gls{HTE} \cite{fairhurst-agosta_multi-physics_2020}.
% Still Hydrogen
Water electrolysis converts electric and thermal energy into chemical energy stored in hydrogen.
The process enthalpy change $\Delta H$ determines the required energy for the electrolysis reaction to take place
\begin{align}
  \Delta H = \Delta G + T \Delta S
\end{align}
where $\Delta G$ is the specific electrical energy and $T \Delta S$ the specific thermal energy.

In LTE, electricity generates the thermal energy.
Hence, $\Delta H$ determines the process’s energy requirement.
$\Delta H$ is equal to 60 kWh/kg-H$_2$ considering a 67\% electrolizer electrical efficiency.

In HTE, a high-temperature heat source is necessary to provide the thermal energy.
Scenario 2 considers HTE as an energy storage mechanism, where a nuclear reactor supplies the thermal energy.
$\Delta G$ decreases with increasing temperatures, as shown in Figure \ref{fig:hte-energy}.
% Decreasing the electricity requirement results in higher overall production efficiencies since heat-engine-based electrical work has a thermal efficiency of 50\% or less.
Decreasing the electricity requirement results in higher overall production efficiencies.

\begin{figure}[htbp!] %or H 
    \centering
    \includegraphics[width=0.90\linewidth]{figures/hte-energy}
    \hfill
    \caption{Energy required by high-temperature electrolysis at atmospheric pressure.}
    \label{fig:hte-energy}
\end{figure}

After converting the excess of energy into hydrogen, it can be stored until the grid requires it.
The conversion of hydrogen produces 40 kWh/kg-H$_2$ \cite{ursua_hydrogen_2012}.
However, conventional fuel cells can use up to 60\% of that energy \cite{doe_energy_fuel_2015}.

% I haven't mentioned that the HTE happens at 3.5 MPa


\section{Methodology}

As mentioned earlier, the focus of this paper is to decrease the CO$_2$ emissions on UIUC campus.
This paper studies 2 energy alternatives, Scenario 1 and 2, and compares them to the default case, Scenario 0.

\textbf{Scenario 0} considers the business as usual case.
We calculate the CO$_2$ emissions using the electricity production hourly data.
These data are available from UIUC Facilities and Services per request.
To narrow down the scope, we focus on months of the two most demanding seasons of the year, January for the winter and June for the summer.
% Abbott and the purchased electricity are estimated to emit 0.26 tCO$_2$/MW(th)h and 0.825 tCO$_2$/MWh, respectively \cite{isee_illinois_2015, isee_illinois_2020}.
% I haven't mentioned that we consider abbot to have a thermal efficiency of 60 \%
% Abbott: 0.26 * 0.907185 / 0.6 /// Purchases: 0.825 * 0.907185
Abbott and the purchased electricity are estimated to emit 0.39 MT CO$_2$/MWh and 0.75 MT CO$_2$/MWh, respectively \cite{isee_illinois_2015, isee_illinois_2020}.

\textbf{Scenario 1} studies the increase of the wind and solar generation capacity up to ten times fold.
The increase in the wind and solar generation capacity requires the addition of electricity storage mechanisms into the mix.
The development of a dispatch model is necessary to calculate the energy being stored, Abbott hourly production, and the purchased electricity.
This model, gives priority to the different sources in the following order: wind and solar, storage mechanism, Abbott, and purchases, as shown in Figure \ref{fig:dispatch-model}.
The \gls{ND} is calculated subtracting the wind and solar generation from the total demand.
Abbott’s electricity generation is preferred over imports because of their lower CO$_2$ emissions.

\begin{figure}[htbp!] %or H 
    \centering
    \includegraphics[width=0.90\linewidth]{figures/dispatch-model}
    \hfill
    \caption{Scenario 1 and 2 dispatch model that calculates the hourly generation of the different sources. \textit{ND} is the net demand.}
    \label{fig:dispatch-model}
\end{figure}

\textbf{Scenario 2} is similar to Scenario 1 but considers the addition of a nuclear reactor into the mix.
This model gives priority to the different sources in the following order: wind and solar, nuclear reactor, storage mechanism, Abbott, and purchases.
The \gls{ND} is calculated subtracting the wind, solar, and nuclear generation from the total demand.
Given that UIUC campus demand is typically smaller than 80 MW, the following analyses consider reactors of small capacities, such as microreactors.
This type of reactors require limited on-site preparation as their components are factory-fabricated and shipped out to the generation site.
Moreover, these reactors use passive safety systems, minimizing electrical parts \cite{us-doe_ultimate_2019}.
% They allow for black starts, being capable of starting up from an utterly de-energized state without receiving power from the grid.
% They can also operate in islanding mode, being able to operate connected to the grid or independently.

Depending on the preferred electrolysis method, the microreactor can supply the hydrogen plant with both electricity and thermal power, as shown in Figure \ref{fig:reactor-hydrogen}, where $\eta$ is the thermal-to-electric conversion efficiency, and $\beta$ and $\gamma$ determine the distribution of the reactor thermal power P$_{th}$ into P$_E$, P$_{EH2}$, and P$_{TH2}$.
$\eta$ is a constant dependent on the reactor outlet temperature.
$\beta$ and $\gamma$ are calculated by the dispatch model.

\begin{figure}[htbp!] %or H 
    \centering
    \includegraphics[width=0.90\linewidth]{figures/reactor-hydrogen}
    \hfill
    \caption{Diagram of a nuclear reactor supplying the grid and a hydrogen plant.}
    \label{fig:reactor-hydrogen}
\end{figure}

This scenario considers 2 reactor power levels, 10 and 20 MW$_{th}$, and 2 reactor outlet temperatures, 300 and 850 $^\circ$C.
Conventional \glspl{LWR} -- with outlet temperatures of 300$^\circ$C -- have efficiencies around 33\% while Gen-IV reactors, for example \glspl{VHTR} -- with outlet temperatures of 850$^\circ$C -- can achieve higher efficiencies around 48\% \cite{fairhurst-agosta_multi-physics_2020}.
Moreover, HTE requires high temperatures which conventional LWRs cannot achieve.
This also means that the HTE hydrogen production is limited by the reactor power, as solar or wind generation only produce electricity.
HTE hydrogen is produced only when the net demand is lower or equal in magnitude than the reactor power.
When the net demand is negative and has a larger magnitude than the reactor power, a secondary storage mechanism becomes necessary.
These case studies use H2-LTE as secondary storage mechanism.


\section{Results}

This section presents and discusses the results for the different scenarios.
% Scenario 0
Table \ref{tab:scenario0} displays the results for Scenario 0.
The emissions in the winter are lower than in the summer due to a lower total demand.
This could be caused by a lower occupancy of campus during January or by the fact that the campus uses mostly steam for heating during the winter.
Given that summer emissions are higher than in the winter, the following analyses focus only in the summer month.

\begin{table}[htbp!]
  \centering
  \caption{Scenario 0 CO$_2$ emissions by source for a month for winter and summer months.}
  \label{tab:scenario0}
  \begin{tabularx}{\textwidth}{@{}*4{>{\hsize=.56\hsize\centering\arraybackslash}X}@{}}
  \toprule
  10$^3$ MT CO$_2$ & Abbott & Purchased & Total \\
  \midrule
  Winter &  6.6 &  8.5 & 15.1 \\
  Summer &  4.8 & 16.7 & 21.5 \\
  \bottomrule
  \end{tabularx}
\end{table}

% Figure - Scenario 1 - Electricity generation distribution
\begin{figure}[htbp!] %or H 
    \centering
    \includegraphics[width=0.90\linewidth]{figures/scenario1-summerB}
    \hfill
    \caption{Scenario 1 hourly generation distribution by source for a capacity increase factor of 10.}
    \label{fig:1-summer-distrib}
\end{figure}

% Scenario 1
Figure \ref{fig:1-summer-distrib} displays an example of the hourly distribution calculated by the dispatch model.
Subtracting the wind and solar generation from the demand allows to calculate the stored electricity, Abbott production, and the purchased electricity.
This figure show that Abbott has enough capacity to supply campus without the need for purchasing electricity.
This study focuses on the reduction of the CO$_2$ emissions and not on the economics.
Eliminating the electricity purchases already reduces the emissions in almost a half, as shown in Figure \ref{fig:1-summer-emissions}.
This figure displays the CO$_2$ emissions as well as wind, solar, and storage capacities for the different capacity increase factors.
Scenario 1 considers a linear increase in the wind and solar capacities.
Hence, the CO$_2$ reduction is also linear as long as no energy storage is required.
The campus wind and solar capacities can be increased up to 25.9 and 14 MW respectively (3 times fold) without a need for installing a storage mechanism.
Once, the energy storage is needed, the higher the charge-discharge efficiency, the larger the reduction is.
This scenario achieves a higher CO$_2$ reduction by means of Li-ion batteries.

% Figure - Scenario 1 - CO2 emissions / Capacities
\begin{figure}[htbp!] %or H 
    \centering
    \includegraphics[width=0.99\linewidth]{figures/scenario1-summerA}
    \hfill
    \caption{Scenario 1 CO$_2$ emissions for different increases in the wind and solar generation capacities.}
    \label{fig:1-summer-emissions}
\end{figure}

% Scenario 2
Figure \ref{fig:2-summer-10-emissions} shows the CO$_2$ emissions and the respective wind, solar, and storage capacities for the different capacity increase factors considering a reactor power of 10 MW$_{th}$.
Once again, the Li-ion batteries show the largest reduction in CO$_2$ emissions.
On the other hand, increasing the solar and wind capacity 10 times fold requires Li-ion batteries with larger capacities than the wind generation or the solar generation, being close to the 100 MW, which is even larger than the campus total demand.

As mentioned earlier, a higher reactor outlet temperature translates into higher efficiencies, and for the same reactor thermal power, a higher electrical output can be achieved.
H2-HTE does not show great advantages over H2-LTE, highlighting that a high outlet temperature is still desirable but only for attaining greater efficiencies.
Additionally, the use of the nuclear reactor reduces the CO$_2$ emissions but does not eliminate them.

% Figure - Scenario 2 - CO2 emissions / Capacities for a 10 MWth-Reactor
\begin{figure}[htbp!] %or H 
    \centering
    \includegraphics[width=0.99\linewidth]{figures/scenario2-10-summer-emissions}
    \hfill
    \caption{Scenario 2 CO$_2$ emissions for different increases in the wind and solar generation capacities, with a 10 MW$_{th}$ reactor.}
    \label{fig:2-summer-10-emissions}
\end{figure}

Figure \ref{fig:2-summer-20-emissions} displays the results for the case considering a reactor of 20 MW$_{th}$.
The Li-ion batteries show the best performance.
A higher reactor power translates into a higher CO$_2$ reduction.
Additionally, a higher reactor power allows for an increase in the hydrogen production, and the H2-LTE and the H2-HTE curves show a wider separation.

% Figure - Scenario 2 - CO2 emissions / Capacities for a 20 MWth-Reactor
\begin{figure}[htbp!] %or H 
    \centering
    \includegraphics[width=0.99\linewidth]{figures/scenario2-20-summer-emissions}
    \hfill
    \caption{Scenario 2 CO$_2$ emissions for different increases in the wind and solar generation capacities, with a 20 MW$_{th}$ reactor.}
    \label{fig:2-summer-20-emissions}
\end{figure}


\section{Conclusions}



%%%%%%%%%%%%%%%%%%%%%%%%%%%%%%%%%%%%%%%%%%%%%%%%%%%%%%%%%%%%%%%%%%%%%%%%%%%%%%%%
\bibliographystyle{ans}
\bibliography{bibliography}
\end{document}

% \begin{figure}[htbp!] %or H 
%     \centering
%     \includegraphics[width=0.95\linewidth]{figures/radial-layout.png}
%     \hfill
%     \caption{Core radial layout. Image reproduced from \cite{oecd_nea_benchmark_2017}.}
%     \label{fig:radial}
% \end{figure}

% \begin{align}
%     \frac{1}{v_g}\frac{\partial}{\partial t} \phi_g &= \nabla \cdot D_g
%     \nabla \phi_g - \Sigma_g^r \phi_g \sum_{g \ne g'}^G
%     \Sigma_{g'\rightarrow g}^s \phi_{g'} \label{eq:diffusion}
%     \intertext{where}
%     C_i &= \mbox{concentration of delayed neutron precursors} \notag \\
%     &\phantom{{}=1} \mbox{in precursor group $i$}.
% \end{align}

% \subsection{Mathematical Basis}
% % Boussinesq approximation
% \begin{align}
% \rho \left[ \frac{\partial \vec{u}}{\partial t} + (\vec{u} \cdot \nabla)\vec{u} \right] &= - \nabla p + \mu \nabla^2 \vec{u} - \rho \vec{g} \\
% \rho &= \rho_0 \left[ 1-\beta(T-T_0) \right] ^ \beta = \frac{1}{T_{ref}} \\
% \rho_0 \left[ \frac{\partial \vec{u}}{\partial t} + (\vec{u} \cdot \nabla)\vec{u} \right] &= - \nabla p + \mu \nabla^2 \vec{u} - \rho_0 \vec{g} \beta (T-T_0)
% \end{align}

